\documentclass{beamer}
\usetheme{Madrid}
\usecolortheme{whale}

\usepackage{amsmath}
\usepackage{amssymb}
\usepackage{amsthm}
\usepackage{mathrsfs}

\title{Mathematical Framework of Quantum Gravity Based on Knot Theory}
\author{Kevin Ting-Kai Kuo}
\date{\today}

\begin{document}

\frame{\titlepage}

\begin{frame}{Outline}
\section{Introduction}
\section{Mathematical Foundation}
\section{Quantum Geometry}
\section{Physical Applications}
\section{Open Questions}
\tableofcontents
\end{frame}

\section{Introduction}
\begin{frame}{Motivation}
\begin{itemize}
\item Challenge: Quantum gravity requires non-perturbative approach
\item Key idea: Use knot theory as mathematical framework
\item Main tools:
    \begin{itemize}
    \item Quantum groups
    \item Spin networks
    \item Loop quantum gravity
    \end{itemize}
\end{itemize}
\end{frame}

\section{Mathematical Foundation}
\begin{frame}{Quantum Groups and 6j-Symbols}
\begin{itemize}
\item Quantum deformation $U_q(\mathfrak{g})$
\item Key structure:
\[
\begin{Bmatrix} 
j_1 & j_2 & j_{12} \\
j_3 & j & j_{23}
\end{Bmatrix}_q
\]
\item Properties:
    \begin{itemize}
    \item Tetrahedral symmetry
    \item Quantum Racah identity
    \end{itemize}
\end{itemize}
\end{frame}

\begin{frame}{Crossing Relations}
\begin{itemize}
\item R-matrix:
\[
R_{j_1j_2} = q^{H_{j_1} \otimes H_{j_2}/2}P_{j_1j_2}
\]
\item Yang-Baxter equation:
\[
R_{12}R_{13}R_{23} = R_{23}R_{13}R_{12}
\]
\item Reidemeister moves invariance
\end{itemize}
\end{frame}

\section{Quantum Geometry}
\begin{frame}{Geometric Operators}
\begin{itemize}
\item Area operator:
\[
\hat{A}_q(S) = l_P^2 \sum_{p \in S \cap \Gamma} \sqrt{[j_p]_q[j_p+1]_q}
\]
\item Volume operator:
\[
\hat{V}_q(R) = l_P^3 \sum_{v \in R} \sqrt{|\hat{Q}_v|}
\]
\item Discrete spectrum
\end{itemize}
\end{frame}

\begin{frame}{Holonomy-Flux Algebra}
\begin{itemize}
\item Holonomy:
\[
h_{\gamma}[A] = \mathcal{P}\exp\int_{\gamma} A
\]
\item Flux:
\[
E(S,f) = \int_S f^i\epsilon_{abc}E^a_idx^b\wedge dx^c
\]
\item Cross relations:
\[
[E(S,f), h_\gamma] = i\hbar\kappa\beta(S,\gamma)X^f h_\gamma
\]
\end{itemize}
\end{frame}

\section{Physical Applications}
\begin{frame}{Quantum Black Holes}
\begin{itemize}
\item Horizon quantum states
\item Area spectrum:
\[
A = 8\pi\gamma l_P^2\sum_i\sqrt{j_i(j_i+1)}
\]
\item Entropy calculation:
\[
S = \frac{A}{4l_P^2} + \text{corrections}
\]
\end{itemize}
\end{frame}

\begin{frame}{Cosmological Applications}
\begin{itemize}
\item Quantum cosmology
\item Big bounce scenario
\item Modified dispersion relations
\item Quantum gravity corrections
\end{itemize}
\end{frame}

\section{Open Questions}
\begin{frame}{Future Directions}
\begin{itemize}
\item Semiclassical limit
\item Matter coupling
\item Observational tests
\item Connection to other approaches
\end{itemize}
\end{frame}

\begin{frame}{Conclusions}
\begin{itemize}
\item Mathematical consistency
\item Physical predictions
\item Experimental challenges
\item Future prospects
\end{itemize}
\end{frame}

\end{document}
